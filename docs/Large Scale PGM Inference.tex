


% Header, overrides base

    % Make sure that the sphinx doc style knows who it inherits from.
    \def\sphinxdocclass{article}

    % Declare the document class
    \documentclass[letterpaper,10pt,english]{/home/londenberg/python-env/clean/lib/python2.7/site-packages/sphinx/texinputs/sphinxhowto}

    % Imports
    \usepackage[utf8]{inputenc}
    \DeclareUnicodeCharacter{00A0}{\\nobreakspace}
    \usepackage[T1]{fontenc}
    \usepackage{babel}
    \usepackage{times}
    \usepackage{import}
    \usepackage[Bjarne]{/home/londenberg/python-env/clean/lib/python2.7/site-packages/sphinx/texinputs/fncychap}
    \usepackage{longtable}
    \usepackage{/home/londenberg/python-env/clean/lib/python2.7/site-packages/sphinx/texinputs/sphinx}
    \usepackage{multirow}

    \usepackage{amsmath}
    \usepackage{amssymb}
    \usepackage{ucs}
    \usepackage{enumerate}

    % Used to make the Input/Output rules follow around the contents.
    \usepackage{needspace}

    % Pygments requirements
    \usepackage{fancyvrb}
    \usepackage{color}
    % ansi colors additions
    \definecolor{darkgreen}{rgb}{.12,.54,.11}
    \definecolor{lightgray}{gray}{.95}
    \definecolor{brown}{rgb}{0.54,0.27,0.07}
    \definecolor{purple}{rgb}{0.5,0.0,0.5}
    \definecolor{darkgray}{gray}{0.25}
    \definecolor{lightred}{rgb}{1.0,0.39,0.28}
    \definecolor{lightgreen}{rgb}{0.48,0.99,0.0}
    \definecolor{lightblue}{rgb}{0.53,0.81,0.92}
    \definecolor{lightpurple}{rgb}{0.87,0.63,0.87}
    \definecolor{lightcyan}{rgb}{0.5,1.0,0.83}

    % Needed to box output/input
    \usepackage{tikz}
        \usetikzlibrary{calc,arrows,shadows}
    \usepackage[framemethod=tikz]{mdframed}

    \usepackage{alltt}

    % Used to load and display graphics
    \usepackage{graphicx}
    \graphicspath{ {figs/} }
    \usepackage[Export]{adjustbox} % To resize

    % used so that images for notebooks which have spaces in the name can still be included
    \usepackage{grffile}


    % For formatting output while also word wrapping.
    \usepackage{listings}
    \lstset{breaklines=true}
    \lstset{basicstyle=\small\ttfamily}
    \def\smaller{\fontsize{9.5pt}{9.5pt}\selectfont}

    %Pygments definitions
    
\makeatletter
\def\PY@reset{\let\PY@it=\relax \let\PY@bf=\relax%
    \let\PY@ul=\relax \let\PY@tc=\relax%
    \let\PY@bc=\relax \let\PY@ff=\relax}
\def\PY@tok#1{\csname PY@tok@#1\endcsname}
\def\PY@toks#1+{\ifx\relax#1\empty\else%
    \PY@tok{#1}\expandafter\PY@toks\fi}
\def\PY@do#1{\PY@bc{\PY@tc{\PY@ul{%
    \PY@it{\PY@bf{\PY@ff{#1}}}}}}}
\def\PY#1#2{\PY@reset\PY@toks#1+\relax+\PY@do{#2}}

\expandafter\def\csname PY@tok@gd\endcsname{\def\PY@tc##1{\textcolor[rgb]{0.63,0.00,0.00}{##1}}}
\expandafter\def\csname PY@tok@gu\endcsname{\let\PY@bf=\textbf\def\PY@tc##1{\textcolor[rgb]{0.50,0.00,0.50}{##1}}}
\expandafter\def\csname PY@tok@gt\endcsname{\def\PY@tc##1{\textcolor[rgb]{0.00,0.27,0.87}{##1}}}
\expandafter\def\csname PY@tok@gs\endcsname{\let\PY@bf=\textbf}
\expandafter\def\csname PY@tok@gr\endcsname{\def\PY@tc##1{\textcolor[rgb]{1.00,0.00,0.00}{##1}}}
\expandafter\def\csname PY@tok@cm\endcsname{\let\PY@it=\textit\def\PY@tc##1{\textcolor[rgb]{0.25,0.50,0.50}{##1}}}
\expandafter\def\csname PY@tok@vg\endcsname{\def\PY@tc##1{\textcolor[rgb]{0.10,0.09,0.49}{##1}}}
\expandafter\def\csname PY@tok@m\endcsname{\def\PY@tc##1{\textcolor[rgb]{0.40,0.40,0.40}{##1}}}
\expandafter\def\csname PY@tok@mh\endcsname{\def\PY@tc##1{\textcolor[rgb]{0.40,0.40,0.40}{##1}}}
\expandafter\def\csname PY@tok@go\endcsname{\def\PY@tc##1{\textcolor[rgb]{0.53,0.53,0.53}{##1}}}
\expandafter\def\csname PY@tok@ge\endcsname{\let\PY@it=\textit}
\expandafter\def\csname PY@tok@vc\endcsname{\def\PY@tc##1{\textcolor[rgb]{0.10,0.09,0.49}{##1}}}
\expandafter\def\csname PY@tok@il\endcsname{\def\PY@tc##1{\textcolor[rgb]{0.40,0.40,0.40}{##1}}}
\expandafter\def\csname PY@tok@cs\endcsname{\let\PY@it=\textit\def\PY@tc##1{\textcolor[rgb]{0.25,0.50,0.50}{##1}}}
\expandafter\def\csname PY@tok@cp\endcsname{\def\PY@tc##1{\textcolor[rgb]{0.74,0.48,0.00}{##1}}}
\expandafter\def\csname PY@tok@gi\endcsname{\def\PY@tc##1{\textcolor[rgb]{0.00,0.63,0.00}{##1}}}
\expandafter\def\csname PY@tok@gh\endcsname{\let\PY@bf=\textbf\def\PY@tc##1{\textcolor[rgb]{0.00,0.00,0.50}{##1}}}
\expandafter\def\csname PY@tok@ni\endcsname{\let\PY@bf=\textbf\def\PY@tc##1{\textcolor[rgb]{0.60,0.60,0.60}{##1}}}
\expandafter\def\csname PY@tok@nl\endcsname{\def\PY@tc##1{\textcolor[rgb]{0.63,0.63,0.00}{##1}}}
\expandafter\def\csname PY@tok@nn\endcsname{\let\PY@bf=\textbf\def\PY@tc##1{\textcolor[rgb]{0.00,0.00,1.00}{##1}}}
\expandafter\def\csname PY@tok@no\endcsname{\def\PY@tc##1{\textcolor[rgb]{0.53,0.00,0.00}{##1}}}
\expandafter\def\csname PY@tok@na\endcsname{\def\PY@tc##1{\textcolor[rgb]{0.49,0.56,0.16}{##1}}}
\expandafter\def\csname PY@tok@nb\endcsname{\def\PY@tc##1{\textcolor[rgb]{0.00,0.50,0.00}{##1}}}
\expandafter\def\csname PY@tok@nc\endcsname{\let\PY@bf=\textbf\def\PY@tc##1{\textcolor[rgb]{0.00,0.00,1.00}{##1}}}
\expandafter\def\csname PY@tok@nd\endcsname{\def\PY@tc##1{\textcolor[rgb]{0.67,0.13,1.00}{##1}}}
\expandafter\def\csname PY@tok@ne\endcsname{\let\PY@bf=\textbf\def\PY@tc##1{\textcolor[rgb]{0.82,0.25,0.23}{##1}}}
\expandafter\def\csname PY@tok@nf\endcsname{\def\PY@tc##1{\textcolor[rgb]{0.00,0.00,1.00}{##1}}}
\expandafter\def\csname PY@tok@si\endcsname{\let\PY@bf=\textbf\def\PY@tc##1{\textcolor[rgb]{0.73,0.40,0.53}{##1}}}
\expandafter\def\csname PY@tok@s2\endcsname{\def\PY@tc##1{\textcolor[rgb]{0.73,0.13,0.13}{##1}}}
\expandafter\def\csname PY@tok@vi\endcsname{\def\PY@tc##1{\textcolor[rgb]{0.10,0.09,0.49}{##1}}}
\expandafter\def\csname PY@tok@nt\endcsname{\let\PY@bf=\textbf\def\PY@tc##1{\textcolor[rgb]{0.00,0.50,0.00}{##1}}}
\expandafter\def\csname PY@tok@nv\endcsname{\def\PY@tc##1{\textcolor[rgb]{0.10,0.09,0.49}{##1}}}
\expandafter\def\csname PY@tok@s1\endcsname{\def\PY@tc##1{\textcolor[rgb]{0.73,0.13,0.13}{##1}}}
\expandafter\def\csname PY@tok@sh\endcsname{\def\PY@tc##1{\textcolor[rgb]{0.73,0.13,0.13}{##1}}}
\expandafter\def\csname PY@tok@sc\endcsname{\def\PY@tc##1{\textcolor[rgb]{0.73,0.13,0.13}{##1}}}
\expandafter\def\csname PY@tok@sx\endcsname{\def\PY@tc##1{\textcolor[rgb]{0.00,0.50,0.00}{##1}}}
\expandafter\def\csname PY@tok@bp\endcsname{\def\PY@tc##1{\textcolor[rgb]{0.00,0.50,0.00}{##1}}}
\expandafter\def\csname PY@tok@c1\endcsname{\let\PY@it=\textit\def\PY@tc##1{\textcolor[rgb]{0.25,0.50,0.50}{##1}}}
\expandafter\def\csname PY@tok@kc\endcsname{\let\PY@bf=\textbf\def\PY@tc##1{\textcolor[rgb]{0.00,0.50,0.00}{##1}}}
\expandafter\def\csname PY@tok@c\endcsname{\let\PY@it=\textit\def\PY@tc##1{\textcolor[rgb]{0.25,0.50,0.50}{##1}}}
\expandafter\def\csname PY@tok@mf\endcsname{\def\PY@tc##1{\textcolor[rgb]{0.40,0.40,0.40}{##1}}}
\expandafter\def\csname PY@tok@err\endcsname{\def\PY@bc##1{\setlength{\fboxsep}{0pt}\fcolorbox[rgb]{1.00,0.00,0.00}{1,1,1}{\strut ##1}}}
\expandafter\def\csname PY@tok@kd\endcsname{\let\PY@bf=\textbf\def\PY@tc##1{\textcolor[rgb]{0.00,0.50,0.00}{##1}}}
\expandafter\def\csname PY@tok@ss\endcsname{\def\PY@tc##1{\textcolor[rgb]{0.10,0.09,0.49}{##1}}}
\expandafter\def\csname PY@tok@sr\endcsname{\def\PY@tc##1{\textcolor[rgb]{0.73,0.40,0.53}{##1}}}
\expandafter\def\csname PY@tok@mo\endcsname{\def\PY@tc##1{\textcolor[rgb]{0.40,0.40,0.40}{##1}}}
\expandafter\def\csname PY@tok@kn\endcsname{\let\PY@bf=\textbf\def\PY@tc##1{\textcolor[rgb]{0.00,0.50,0.00}{##1}}}
\expandafter\def\csname PY@tok@mi\endcsname{\def\PY@tc##1{\textcolor[rgb]{0.40,0.40,0.40}{##1}}}
\expandafter\def\csname PY@tok@gp\endcsname{\let\PY@bf=\textbf\def\PY@tc##1{\textcolor[rgb]{0.00,0.00,0.50}{##1}}}
\expandafter\def\csname PY@tok@o\endcsname{\def\PY@tc##1{\textcolor[rgb]{0.40,0.40,0.40}{##1}}}
\expandafter\def\csname PY@tok@kr\endcsname{\let\PY@bf=\textbf\def\PY@tc##1{\textcolor[rgb]{0.00,0.50,0.00}{##1}}}
\expandafter\def\csname PY@tok@s\endcsname{\def\PY@tc##1{\textcolor[rgb]{0.73,0.13,0.13}{##1}}}
\expandafter\def\csname PY@tok@kp\endcsname{\def\PY@tc##1{\textcolor[rgb]{0.00,0.50,0.00}{##1}}}
\expandafter\def\csname PY@tok@w\endcsname{\def\PY@tc##1{\textcolor[rgb]{0.73,0.73,0.73}{##1}}}
\expandafter\def\csname PY@tok@kt\endcsname{\def\PY@tc##1{\textcolor[rgb]{0.69,0.00,0.25}{##1}}}
\expandafter\def\csname PY@tok@ow\endcsname{\let\PY@bf=\textbf\def\PY@tc##1{\textcolor[rgb]{0.67,0.13,1.00}{##1}}}
\expandafter\def\csname PY@tok@sb\endcsname{\def\PY@tc##1{\textcolor[rgb]{0.73,0.13,0.13}{##1}}}
\expandafter\def\csname PY@tok@k\endcsname{\let\PY@bf=\textbf\def\PY@tc##1{\textcolor[rgb]{0.00,0.50,0.00}{##1}}}
\expandafter\def\csname PY@tok@se\endcsname{\let\PY@bf=\textbf\def\PY@tc##1{\textcolor[rgb]{0.73,0.40,0.13}{##1}}}
\expandafter\def\csname PY@tok@sd\endcsname{\let\PY@it=\textit\def\PY@tc##1{\textcolor[rgb]{0.73,0.13,0.13}{##1}}}

\def\PYZbs{\char`\\}
\def\PYZus{\char`\_}
\def\PYZob{\char`\{}
\def\PYZcb{\char`\}}
\def\PYZca{\char`\^}
\def\PYZam{\char`\&}
\def\PYZlt{\char`\<}
\def\PYZgt{\char`\>}
\def\PYZsh{\char`\#}
\def\PYZpc{\char`\%}
\def\PYZdl{\char`\$}
\def\PYZhy{\char`\-}
\def\PYZsq{\char`\'}
\def\PYZdq{\char`\"}
\def\PYZti{\char`\~}
% for compatibility with earlier versions
\def\PYZat{@}
\def\PYZlb{[}
\def\PYZrb{]}
\makeatother


    %Set pygments styles if needed...
    
        \definecolor{nbframe-border}{rgb}{0.867,0.867,0.867}
        \definecolor{nbframe-bg}{rgb}{0.969,0.969,0.969}
        \definecolor{nbframe-in-prompt}{rgb}{0.0,0.0,0.502}
        \definecolor{nbframe-out-prompt}{rgb}{0.545,0.0,0.0}

        \newenvironment{ColorVerbatim}
        {\begin{mdframed}[%
            roundcorner=1.0pt, %
            backgroundcolor=nbframe-bg, %
            userdefinedwidth=1\linewidth, %
            leftmargin=0.1\linewidth, %
            innerleftmargin=0pt, %
            innerrightmargin=0pt, %
            linecolor=nbframe-border, %
            linewidth=1pt, %
            usetwoside=false, %
            everyline=true, %
            innerlinewidth=3pt, %
            innerlinecolor=nbframe-bg, %
            middlelinewidth=1pt, %
            middlelinecolor=nbframe-bg, %
            outerlinewidth=0.5pt, %
            outerlinecolor=nbframe-border, %
            needspace=0pt
        ]}
        {\end{mdframed}}
        
        \newenvironment{InvisibleVerbatim}
        {\begin{mdframed}[leftmargin=0.1\linewidth,innerleftmargin=3pt,innerrightmargin=3pt, userdefinedwidth=1\linewidth, linewidth=0pt, linecolor=white, usetwoside=false]}
        {\end{mdframed}}

        \renewenvironment{Verbatim}[1][\unskip]
        {\begin{alltt}\smaller}
        {\end{alltt}}
    

    % Help prevent overflowing lines due to urls and other hard-to-break 
    % entities.  This doesn't catch everything...
    \sloppy

    % Document level variables
    \title{Large Scale PGM Inference}
    \date{June 7, 2014}
    \release{}
    \author{Unknown Author}
    \renewcommand{\releasename}{}

    % TODO: Add option for the user to specify a logo for his/her export.
    \newcommand{\sphinxlogo}{}

    % Make the index page of the document.
    \makeindex

    % Import sphinx document type specifics.
     


% Body

    % Start of the document
    \begin{document}

        
            \maketitle
        

        


        
        \subsubsection{Scalable Nonparametric Directed Graphical Model Inference
and Learning}

Author: Kai Londenberg (Kai.Londenberg@gmail.com), June 2014

\paragraph{Abstract}

This short article tries to give an overview over complementary
techniques over MCMC for general inference in arbitrary directed
probabilistic graphical models. The focus lies on techniques and
algorithms for creating hybrid models which can be scaled to high

dimensional problems, problems with huge data sets and distributed among
multiple machines.

\paragraph{Motivation}

Two papers got me thinking:
\href{http://www.dauwels.com/Papers/Particle.pdf}{Dauwels et al:
Particle Methods as Message Passing}, which gives a nice overview of how
to generalize Message Passing methods, by mixing sampling based methods
freely with exact or fast approximate inference algorithms.

The second is \href{http://arxiv.org/pdf/1311.4780v1.pdf}{Neiswanger et
al: Embarassingly Parallel MCMC}, where an algorithm is described which
can be used to scale MCMC to problems with huge data sets.

Both algorithms actually suffer from the same problem, namely the
\textbf{Message Fusion} problem (described way down). A problem which
has luckily been successfully solved before. I add to that list by
proposing a new approach to efficient density estimation from MCMC
models, called \textbf{Density Mapping}

What I hope is, that these ideas can lead to a practical implementation
of a general, flexible inference system with semantics similar to those
found in common MCMC packages for Bayesian Inference, but with the
capability to outperform these for problems with huge data sets or a
large number of dimensions if the distribution can be factorized into
smaller problems somehow.

I hope to be able to extend the existing Bayesian Modeling Toolkit for
Python \href{https://github.com/pymc-devs/pymc}{PyMC3} via my
side-project \href{https://github.com/kadeng/pypgmc}{PyPGMc} to support
the algorithms mentioned in this article.

So this article is both an overview, and sort of a collection of ideas
and roadmap items.\paragraph{Introduction to Directed Graphical Models}

Directed Probabilistic Graphical Models (DGMs) provide a flexible and
powerful framework for probabilistic reasoning. In all generality, they
are a way to efficiently represent complex probability distributions in
high dimensional spaces by factorizing the joint distribution into
conditional distributions.

Given a set of random variables $x_i \sim X_i$ and their joint random
vector $x \sim X$ with $x = \{ x_1, \ldots, x_N \}$ we represent their
joint distribution as the product of a set of conditional distributions

\[
P(X) = \prod_i^N P_i(X_i|{pa}(X_i) 
\]

Where $pa(X_i)$ is the set of parents of variable $X_i$ in a directed
graph $G$, where each vertex in the graph represents one of the random
variables.Each vertex in this graph could assign an arbitrary probability
distribution to it's random variables. But this probability distribution
may only depend on the parents of the variable in the graph.

Such a representation has many advantages, even to list them completely
would be out of the scope of this article. An excellent overview is
present in the book
\href{http://mitpress.mit.edu/books/probabilistic-graphical-models}{Probabilistic
Graphical Models} by Daphne Koller, who also offers a
\href{https://www.coursera.org/course/pgm}{Coursera course} by the same
name, which is highly recommended.\paragraph{D-Separation / Conditional Independence}

Most importantly, the graph encodes information about (conditional)
independencies among the variables. By a property called D-Separation
which can be determined using a few simple rules, we can safely
determine whether the probability distribution of a given set of
variables in the graph can be affected by changes in the probability of
another set of variables \textbf{given} another set of variables which
are held fixed.

If the conditional dependencies that hold over a joint probability
distribution are a subset of the conditional independence assumptions
made by the graph, this distribution is compatible to the graph in the
sense that the distribution can be faithfully represented by a
factorization of the distribution along that graph.\paragraph{Causal Models}

A very common form of these models is to restrict parent / child
relationships to cause/effect pairs. Furthermore, the network must be
complete in the sense that no common causes of any two variables are
missing from the model.

While it is not neccessary for the machinery of DGMs to work that they
are causal models, this can, under certain circumstances, be used to
perform so called causal inference or causal reasoning using DGMs. More
on that can be found in Judea Pearl's excellent book on
\href{http://bayes.cs.ucla.edu/BOOK-2K/}{Causality}.

One important rule to note is, that in order to perform causal inference
in such a causal network, i.e.~to estimate the impact of an explicit
\textbf{action} where the value of a variable is forced to have a
certain value (in contrast to observing it having that value) it is
neccessary to sever all ties from the parents of said variable to it
(since they are no longer causally connected). Furthermore, the network

In Pearl's see/do calculus, he discerns between $P(x|y)$ ( probability
of x given that I \emph{see} $y$ ) and $P(x|do(y))$ ( probability of x
given that I \emph{do} x).

While not of further concern here, Pearl provides a great deal of
informal insights and formal rules into when and how observations can be
converted into causal claims, how to transfer the results of studies
from one setting to another.\paragraph{Common types of DGMs}

Some common types of DGMs that you might have heard about include:

\begin{itemize}
\item
  Bayesian Networks (BNs)
\item
  Hierarchical Bayesian Models
\item
  (Gaussian) Mixture Models (GMMs)
\end{itemize}
Also a lot of common models for time-series can be thought of as DGMs,
among them:

\begin{itemize}
\item
  Vector Auto-Regressive Models (VAR)
\item
  Hidden Markov Models (HMMs)
\item
  State Space Models (SSM)
\item
  Dynamic Bayesian Networks (DBNs)
\end{itemize}
Many of these models have their own set of specialized inference and
learning algorithms, their own set of advantages and disadvantages.\paragraph{Inference / Reasoning in Graphical Models}

Generally speaking, we can use these graphical models to reason about
the marginal probability distributions, most likely configurations (MLE
/ MAP configurations) etc. of variables of interest \textbf{given
evidence}. This in turn can be used in many applications, from decision
support (making decisions under uncertainty) and as a key component for
supervised, unsupervised and semi-supervised learning.\paragraph{Parametric VS Nonparametric representations}

If we can restrict our probability distributions to come from specifiy
families of distributions, inference can be made very efficient in some
cases. But if you want to have a general model which can capture any
kind of weird multi-modal and non-continous distributions, you are
limited to very slow inference using MCMC methods. Also, the scalability
in these cases is very limited, because most MCMC algorithms are not
made to be distributed.\textbar{} Family \textbar{} Advantages \textbar{} Disadvantages
\textbar{}
\textbar{}------------------------------------------\textbar{}----------------\textbar{}--------------------------------------------------------------------------------------------------------------------------------------------------------------------------\textbar{}
\textbar{} Conditional Linear Gaussian (CLG) \textbar{} Very Fast
\textbar{} Are your distributions linear combinations of gaussians ?
\textbar{} \textbar{} Discrete (Categorical) \textbar{} Fast \textbar{}
High number of parameters if number of parents of any variable, or
number of discrete ``bins'' of variables becomes too large. Quickly
becomes intractable in these cases. \textbar{} \textbar{} Generic
(arbitrary functions of parents) \textbar{} Most Flexible \textbar{}
Very slow inference (MCMC) or strongly biased approximate inference,
hard to determine convergence / mixing. Usually intractable in high
dimensions. \textbar{}\paragraph{Hierarchical Bayesian Models and MCMC}

In Bayesian Hierarchical Modeling, we are usually either interested in
estimating marginals of certain model parameters in order to gain
insights into specific problems, or we are interested in evaluating the
expected value of some (utility- or loss-) function over the posterior
of a set of random variables.

Given that in the Bayesian view, the unknown parameters of a model are
random variables like any other, so we can use the machinery of DGM
inference. Since these models can have almost arbitrary functional
relations between variables, it is commong to perform Markov Chain Monte
Carlo simulation to sample from the posterior distribution.

What is problematic about these methods is that inference is usually
slow, and it is hard to determine whether the model has converged
(mixed) to a stable posterior. Generally, MCMC does not scale well to
high-dimensional problems using established methods (yet), despite the
fact that there have been some special areas where MCMC methods could be
applied to solve high dimensional problems such as large scale matrix
factorization for recommender systems.

MCMC, while an approximate approach, is asymptotically exact if applied
correctly.\paragraph{Message Passing Algorithms}

Among the most efficient algorithms for exact and approximate inference
in discrete and conditional linear gaussian DGMs are so called Message
Passing (MP) or Belief Propagation Algorithms. These algorithms operate
on a so-called factor graph, which is very similar to a DGM, except that
vertices (factors) may represent joint distributions of multiple
variables. If factors share variables, they have to be connected (at
least indirectly) using a chain of factors where each factor contains
that variable.

Correspondingly, edges (along which messages are passed) need to be able
to convey information about joint distributions of several variables at
once.

By collapsing a DGM into a factor graph tree (so called Clique Tree or
Junction Tree), it is possible to perform efficient exact inference on
discrete and conditional linear gaussian networks using message passing
inference. That is, unless the resulting tree has at some point a too
large tree-width (loosely, the result of a large but too dense graph),
which can make exact inference intractable.

Even in those cases where exact inference is intractable, the Loopy
Belief Propagation algorithm can provide very fast approximate
(asymptotically biased) inference, providing good solutions (empirical
results) in cases where other algorithms fail or are too slow.

It is important to note that the core algorithm (message passing) of
approximate Loopy Belief Propagation and exact Clique Tree Inference are
the same.

Again, I refer to the book
\href{http://mitpress.mit.edu/books/probabilistic-graphical-models}{Probabilistic
Graphical Models} by Daphne Koller and her
\href{https://www.coursera.org/course/pgm}{Coursera course} for details.\paragraph{Nonparametric and Particle Belief Propagation / Message
Passing}

As Dauwels et al. have pointed out in the Paper
\href{http://www.dauwels.com/Papers/Particle.pdf}{Particle Methods as
Message Passing}, it is actually possible to combine parametric exact
inference and nonparametric approximate inference by using
\textbf{Particle Lists} as messages in the message passing algorithm.

They used this approach to show that it is possible to view common MCMC
procedures such as Gibbs Sampling, Metropolis Hastings and Importance
Sampling as special cases of \textbf{Particle Message Passing}.

What is also important here: Each factor \textbf{could be an independend
MCMC sampler working on a subset of the variables and / or evidence}. Or
a faster parametric inference algorithm, if the problem allows.

Other fast inference algorithms such as \textbf{Expectation Propagation}
and other Forms of Variational Inference such as \textbf{Mean-Field}
based method can be also be cast as variants of Message Passing
procedures.

There are several key papers which describe important aspects and
approaches that might be taken:

\begin{itemize}
\item
  \href{http://www.dauwels.com/Papers/Particle.pdf}{Dauwels et Al:
  Particle Methods as Message Passing}
\item
  \href{http://ssg.mit.edu/nbp/papers/nips03.pdf}{Ihler, Sudderth et al:
  Nonparametric Belief Propagation}
\item
  \href{http://ssg.mit.edu/nbp/papers/nips03.pdf}{Ihler, Sudderth et al:
  Efficient Multiscale Sampling from Products of Gaussian Mixtures}
\item
  \href{http://machinelearning.wustl.edu/mlpapers/paper\_files/AISTATS09\_IhlerM.pdf}{Ihler,
  Mc. Allister: Particle Belief Propagation}
\item
  \href{http://robotics.stanford.edu/~koller/Papers/Koller+al:UAI99.pdf}{Koller
  et al: A General Algorithm for Approximate Inference and Its
  Application to Hybrid Bayes Nets}
\end{itemize}\subparagraph{The Message Fusion Problem}

All of the above papers identify a single performance bottleneck in
these algorithms.

If we have two factors which share at least one continuous variable $x$
with a smoot pdf, the probability that two factors will independently
choose the same value for that variable is essentially zero.

So if we have two factors represented using discrete particle lists
$\phi(x)$ and $\theta(x)$, their product will be zero with near
certainty everywhere.

What we need to do is to perform so called \textbf{Message Fusion}, for
which several approaches have been proposed.

The standard approach is to use some form of \textbf{Kernel Density
Estimation} (KDE), which effectively represents each particle/sample not
as a discrete probability spike, but smoothes the density using an
appropriate kernel. Usually, Gaussian Kernels are used. Given an
efficient sampling procedure from products of Gaussian Mixtures, such as
\href{http://ssg.mit.edu/nbp/papers/nips03.pdf}{Ihler, Sudderth et al:
Efficient Multiscale Sampling from Products of Gaussian Mixtures}, we
can efficiently sample from these. More on that approach is found in
\href{http://ssg.mit.edu/nbp/papers/nips03.pdf}{Ihler, Sudderth et al:
Nonparametric Belief Propagation}

But a problem remains: It's computationally expensive to evaluate the
probability density and curvature (Jacobian and Hessian) of these
messages. And that might be important if we would like to sample from a
product of such a mixture density message with the probability density
function of a factor.\paragraph{Particle Belief Propagation Approach}

In
\href{http://machinelearning.wustl.edu/mlpapers/paper\_files/AISTATS09\_IhlerM.pdf}{Ihler,
Mc. Allister: Particle Belief Propagation} it has been proposed to
sample from the particles in the messages themselves. This is similar to
what happens in \textbf{Particle Filters}. While potentially a good
idea, it (like Particle Filters) suffers from the \emph{thinning}
problem: If the message and the factor do not agree, the number of
useable particles gets very low and the procedure produces unreliable
and/or inaccurate results.

Like with Particle Filters, one approach to fix this problem is to use
\textbf{resampling}. That is, loosely speaking, we tell the original
factor where we got the message from, that we would like to have samples
of a finer resolution in certain regions. Then the original factor
replies with a new (importance sampled) message list, where it provides
new samples, with more (but downweighted) samples in the corresponding
regions of interest.

This procedure can already provide accurate distributed inference. It
just has one problem: It is probably pretty slow (all this re-sampling)
and requires lots of communication.

If we let the numbers of particles in each list become low, it can be
seen as a form of Gibbs Sampling where we exchange not just one particle
(the sample), but multiple of them.

This procedure, as inefficient as it might seem, might have a distinct
advantage over most MCMC Algorithms: \textbf{It allows for much easier
convergence diagnostics}. By measuring the convergence on a per-message
level, we can probably automatically determine when the algorithm has
converged to a final solution, given that we can determine this for each
factor individually.

While this sounds not so much of a great deal, actually it is: For MCMC
you usually need a human expert who decides if the numbers of samples
have been sufficient, if all relevant states have been visited etc. But
even then, that person can never be sure. Much the less, if the problem
gets high-dimensional. Having a clear convergence diagnostic opens the
door for novel applications in large scale risk analysis.\paragraph{Compact Message Density Estimation for Message Fusion}

Another approach, which has also been taken or at least proposed by
several researchers ( see
\href{http://machinelearning.wustl.edu/mlpapers/paper\_files/AISTATS09\_IhlerM.pdf}{Ihler,
Mc. Allister: Particle Belief Propagation} for an overview) is to try to
estimate message densities using some form of nonparametric density
estimation technique which both smoothes the distribution, and
compresses the amount of data required to transfer the message. See
\href{http://robotics.stanford.edu/~koller/Papers/Koller+al:UAI99.pdf}{Koller
et al: A General Algorithm for Approximate Inference and Its Application
to Hybrid Bayes Nets} for a more thorough discussion of this.

In that paper, Density Estimation Trees (DETs) with GMMs at the leaves
have been used with success by Koller et. al. as density estimators, so
that might be a good choice to make as well. They iteratively refined
these density estimates using an iterative approach, similar to the
resampling mentioned above.

Generally, Multivariate Gaussian Mixture Models (GMMs) trained with
Regularized Expectation Maximization (EM) might be a another good
choice. See \href{http://www.cs.ubc.ca/~murphyk/MLbook/}{Kevin Murphy:
Machine Learning: A Probabilistic Perspective, Chapter 11}. These would
lend themselves to the fast methods in
\href{http://ssg.mit.edu/nbp/papers/nips03.pdf}{Ihler, Sudderth et al:
Efficient Multiscale Sampling from Products of Gaussian Mixtures}

But how do we determine the optimal number of mixture components ? Maybe
we can make the algorithm automatically choose the number of components
based on the data ?

One obvious but very slow approach would be to use cross-validation to
select an optimal number of components. But this gets prohibitevely
slow. A different approach would be to use a
\href{http://www.gatsby.ucl.ac.uk/~edward/pub/inf.mix.nips.99.pdf}{Dirichlet
Process Clustering}, also called \textbf{Infinite Gaussian Mixture
Model} to choose a data-dependent number of components. Alternatively, a
possibly better alternative is not to use the Dirichlet Process Prior
for the number of components, but rather a
\href{http://en.wikipedia.org/wiki/Pitman\%E2\%80\%93Yor\_process}{Pitman-Yor
Process}, a more flexible two-parameter generalization of the Dirichlet
Process which allows for Power-Law (fat) tails.

There are a lot of ready-made implementations of these (except for the
Pitman-Yor Process Clustering), see
\href{http://scikit-learn.org/stable/modules/mixture.html}{Scikit-Learn
Documentation: Mixtures}

Another (novel) approach is the following, which might be more efficient
if the Particle Lists are created using MCMC Sampling.\paragraph{Embarassingly Parallel MCMC}

In a recent paper (
\href{http://arxiv.org/pdf/1311.4780v1.pdf}{Neiswanger et al:
Embarassingly Parallel MCMC} ) an asymptotically exact algorithm for
performing embarassingly parallel distributed MCMC was presented.
Interestingly, the main problem solved in that paper is almost exactly
the Message Fusion problem stated above. So by solving one problem, we
get to solve inference for both high-dimensional and big-data problems.\paragraph{Density Mapping (DRAFT)}

The core idea for Density Mapping is to combine MCMC, Gradient Ascent or
EM and Kernel Density Estimation (KDE) into a single, more efficient
algorithm for Kernel Density Estimation which can be used in the context
of large scale distributed Nonparametric Message Passing inference
engines.

We sample from a density function, and then modify the sampling density
function by subtracting from it density estimates around local modes.
This way, the probability density gets simultaneously \emph{mapped out},
ensuring that the MCMC chain spends it's computational time efficiently
by mapping so far uncovered regions of the probability space.

Let $f^*: \mathbb{R}^D \mapsto \mathbb{R}$ be our unnormalized posterior
density function which we can evaluate at any point. The corresponding
normalized density is $P^*$ with $P^*(x) = \frac{1}{Z} f^*(x)$ and
$x \in \mathbb{R}^D$, with $Z$ being the normalization constant, i.e.
$Z = \int{f^*(x) dx}$

We assume that we can consistently estimate the density $P^*$ (usually a
posterior) using a suitable Markov Chain Monte Carlo algorithm such as a
Metropolis Hastings sampler given the unnormalized density function
$f^*(x)$Now let us assume we have a kernel probability density estimate which
has been estimated step by step from N point probability masses:

\[ 
P^{E}(x) = \frac{1}{H} \sum_{i=1}^{N} \gamma_i \cdot K_i(x)
\]

Where $H = \sum{\gamma_i}$ and each $K_i$ is itself a properly
normalized density function (Kernel) with a single mode
$k_i = {argmax}_{x} K_i(x)$ with $k_i \in \mathbb{R}^D$. We define the
unnormalized kernel density at time step N as

\[
f_N^E(x) =  \sum_{i=1}^{N} \frac{f_i^E(k_i)}{K_i(k_i)} \cdot K_i(x)
\]

with $f_0^E(x) = f^*(x)$. Correspondingly, we define
$\gamma_i = \frac{f^*(k_i)}{f_i^E(k_i)}$ which ensures that
$f_N^E(x) = P^{E}(x) = f^*(x)$ for all points
$x \in \{ k_1, \ldots, k_N \}$.

We define the \textbf{Unnormalized Sampling Function} as:

\[
f^F(x) = min(max(f^*(x)-f^E(x), f^*(x)^\frac{1}{s}), f^*(x))
\]

With $s > 1$ being a cooling factor which flattens the original
distribution. Plausible initial values for s might be in the range from
2 to 100 depending on how flat we would like the distribution to become.

If we chose the density estimation Kernels $K_i$ such that they (at
least approximately) have limited support around their mode or mean
$x_i$, this Sampling Function can be calculated (or approximated)
quickly, even if N (the number of kernels in the density estimate) is
large, by just taking the nearest kernels to a given point into account.
Such a lookup can be made efficient using \textbf{KD-Trees} or
\textbf{Cover-Trees} to be performed in $O(D \cdot \log(N))$ time, with
$N$ being the number of components, and $D$ being the dimensionality of
the points.

During or after MCMC sampling, we should check for each sample whether
$f^*(x)-f^E(x)$ becomes negative at that point. This would indicate
regions where we over-estimate the density. In such a case, it should be
possible to shrink the variance of the responsible component of
f\^{}E(x) in that direction. Since this is computationally intensive (we
have to re-compute all mixture components), this should be prevented by
scaling down the variance of the kernel components in directions of high
variance.\paragraph{Density Mapping Algorithm (DRAFT)}

Now, after initializing s to a sensible value, and starting with $f^E=0$
(constant), the density estimation procedure works like this:

\begin{enumerate}[1.]
\item
  Run MCMC to collect a number of samples from $f^F$ . Discard burn-in
  samples.
\item
  Check all sampled points for values with negative
  f\textsuperscript{*(x)-f}E(x). Shrink the variance of responsible
  components of $f^E$.
\item
  Pick a random sample, and perform gradient ascent or EM to find a
  local optimum/mode of $f^F$: called $k_i$
\item
  Check if this is a new local optimum. If not: Increase $s$ and
  continue at 1.)
\item
  Create a Kernel density estimate $K_i$ around local optimum $k_i$ (use
  the Hamiltonian of $f^*$ at $k_i$ as a scale/precision matrix, apply
  sensible regularization)
\item
  (Optional): Add $k_i$ to a KD-Tree index to speed up nearest-neighbour
  lookups
\item
  Update $f^E$ and $f^F$ using the new kernel estimate $K_i$
\item
  Stopping criterion: Has the density been flattened enough ? Then stop.
\item
  otherwise go to 3.) or 1.)
\end{enumerate}
The result (so I hope) is a rather good density estimate. This algorithm
has yet to be tried in practice.\subparagraph{Conclusions}

It seems like everything is ready to build a generic framework for large
scale inference in directed probabilistic graphical models. Someone just
has to do it ..
        

        \renewcommand{\indexname}{Index}
        \printindex

    % End of document
    \end{document}


